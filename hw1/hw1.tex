\documentclass[12pt,letterpaper]{article}
\usepackage{fullpage}
\usepackage[top=2cm, bottom=4.5cm, left=2.5cm, right=2.5cm]{geometry}
\usepackage{amsmath,amsthm,amsfonts,amssymb,amscd}
% \usepackage{lastpage}
\usepackage{enumerate}
\usepackage{fancyhdr}
% \usepackage{mathrsfs}
\usepackage{xcolor}
% \usepackage{graphicx}
\usepackage{listings}
\usepackage{hyperref}

% define vector
\newcommand{\q}{\underline}

% define in-line code
\definecolor{codegray}{gray}{0.9}
\newcommand{\code}[1]{\colorbox{codegray}{\texttt{#1}}}

\hypersetup{%
  colorlinks=true,
  linkcolor=blue,
  linkbordercolor={0 0 1}
}

\renewcommand\lstlistingname{Snippet}
\renewcommand\lstlistlistingname{Snippet}
\def\lstlistingautorefname{Snippet.}

\lstdefinestyle{python}{
    language        = python,
    frame           = lines, 
    basicstyle      = \footnotesize,
    keywordstyle    = \color{violet},
    stringstyle     = \color{green},
    commentstyle    = \color{red}\ttfamily
}


\setlength{\parindent}{0.0in}
\setlength{\parskip}{0.05in}

% Edit these as appropriate
\newcommand\course{Phys 214}
\newcommand\hwnumber{1}                  % <-- homework number
\newcommand\NetIDa{Ming-Feng Ho}           % <-- NetID of person #1
% \newcommand\NetIDb{netid12038}           % <-- NetID of person #2 (Comment this line out for problem sets)

\pagestyle{fancyplain}
\headheight 35pt
\lhead{\NetIDa}
% \lhead{\NetIDa\\\NetIDb}                 % <-- Comment this line out for problem sets (make sure you are person #1)
\chead{\textbf{\Large Homework \hwnumber}}
\rhead{\course \\ \today}
\lfoot{}
\cfoot{}
\rfoot{\small\thepage}
\headsep 1.5em

\begin{document}

\section*{Problem 1}

So let me list some of the values here:
\begin{itemize}
    \item $ (R_* \mid X= 1 , m=20, MK
         ) = 1890 \, \mathrm{e^- / s}$
    
    \item $ (m_{sky} \mid MK, m = m_{sky}, r_{pix}) = 20.3 \, 
    \mathrm{mag/arsec}^2 $
    
    \item $ (r_{pix} \mid MK) = 0.135 \, 
    \mathrm{arcsec/pix} $

    \item $ (RN^2 \mid MK) = 5 \, \mathrm{e^-}$
    \item $ (G \mid MK) = 1.2 \, \mathrm{e^-/DN}$
\end{itemize}

Here are the questions,

\begin{enumerate}
    \item {\bf $R_{sky}$ of detected in R-band?}
    
    We knew, 
    \begin{equation}
        (R_* \mid X= 1 , m=20, MK) = 1890 \, \mathrm{e^- / s}.
    \end{equation}
    Now we took off the condition on $R$ magnitude,
    \begin{equation}
        (R_{*} \mid X=1, m, MK) = 1890 \times 
            10^{ - \frac{m - 20}{ 2.5 } }
    \end{equation}
    since $R_{*} \propto 10^{- \mathrm{mag} / 2.5}$.

    Condition on pixel size to get $R_{sky}$ 
    \begin{equation}
        \begin{split}
        (m_{sky} \mid MK, m = m_{sky},) &=  
            (m_{sky} \mid MK, m = m_{sky}, r_{pix}) - 2.5\log_{10} (r_{pix} \mid MK)^2\\
            & = 20.3 - 5\times\log_{10}0.135\\
            &= 24.6 \, \mathrm{mag/pixel}
        \end{split}
    \end{equation}

    Use the information of sky magnitude on a pixel to get the 
    rate of detection on a pixel, which is
    \begin{equation}
        \begin{split}
            (R_{*} \mid X=1, m=m_{sky}, MK) 
                &= 1890 \times 10^{ - \frac{m_{sky} - 20}{ 2.5 } }\\
                &= 1890 \times 10^{- ( 24.6 - 20 ) / 2.5} = 26.13 \, \mathrm{e^-/pixel/s}
        \end{split}
        \qed
    \end{equation}


    \item {\bf Find $ R_* \mid m=26, X=1.2 $.}

    As suggested by the instructor of fundamental astrophysics- never use
    magnitude,

    \begin{equation}
        \begin{split}
        (R_{*} \mid X=1.2, m=26) 
            & \propto 10^{- \frac{26 - 20}{2.5}} \times \exp{(-cX)}\\
            & \propto 10^{- \frac{26 - 20 + kX}{2.5} }\\
            & \propto 10^{- \frac{26 - 20 + 0.1 * 1.2}{2.5} }.
        \end{split}
    \end{equation}

    Remember to bring back your constant
    \begin{equation}
        (R_{*} \mid X=1.2, m=26) = 1890
            \times 10^{ - \frac{26 - 20 + 0.1 * 1.2}{2.5} } 
            = 6.74 \, \mathrm{e^-/s}.   
    \end{equation}


    \item What's the time range for sky-dominated? Condition on radius of 7 pixels aperture.
    
    Since we have more electrons from sky, we need to wait for enough time
    to reach source-dominated.
    The effect of sky decrease as $t^{-1/2}$ because there's a t in the 
    nominator in S/N equation while there's only $t^{1/2}$ in the 
    denominator.
    
    The uncertainty term came from Poisson distribution, which is
    \begin{equation}
        \sigma = \sqrt{  
            R_* t + R_{sky} n_{pix} t + (RN)^2 n_{pix} + DN \times n_{pix} t
        }
    \end{equation}

    Though the description in the question is unclear about `7' pixels,
    (Does it mean it contains 7 pixels or its radius has a range of 7 pixelscales?)
    I tried both cases and I thought it should be $n_{pix} = 49$. 
    
    Therefore,
    \begin{equation}
        \begin{split}
            \sigma &= \sqrt{  
                R_* t + R_{sky} n_{pix} t + (RN)^2 n_{pix} + DN \times n_{pix} t
            }\\
                &= \sqrt{
                6.74t + 26.13*49*t + 5*49 + 1.2*49*t.
            }
        \end{split}
    \end{equation}

    What we want is sky-dominated, so $\sigma > R_* t = \mathrm{signal}$.
    Allow me to make some sloppy approximations,    
    \begin{equation}
        \begin{split}
            \sigma &> 6.7 t\\
            \sqrt{6.74t + 26.13*49*t + 5*49 + 1.2*49*t} &> \\
            \sqrt{7t + 25*50*t + 250 + 70*t} &\gtrsim \\
            \sqrt{1327*t + 250 } &\gtrsim \\
        \end{split}
    \end{equation}

    Square both sides,
    \begin{equation*}
        \begin{split}
        1327t + 250         &\gtrsim 40 t^2 \\
            \Rightarrow 6   &\gtrsim t^2 - 30t = (t - 15)^2 - 225\\
            \Rightarrow 230 &\gtrsim (t - 15)^2.
        \end{split}
    \end{equation*}

    So roughly before $15 + 15$ seconds is sky-dominated,
    \begin{equation}
        t \lesssim 30 \, \mathrm{s}.
    \end{equation}


    \item {\bf Explain how S/N scale with seeing?}
    
    Seeing is cased by the turbulence in the atmosphere.
    Therefore, the effect of seeing will contribute to the width of PSF.
    To a first order approximation, let's say $\mathrm{seeing} \propto \sigma_{PSF}$.

    So that we arrive this scaling relation,
    \begin{equation}
        n_{pix} \propto r_{measure}^2 
            \propto \sigma_{PSF}^2 
            \propto \mathrm{seeing}^2.
    \end{equation}

    Recall the equation of S/N,
    \begin{equation}
        S/N = \frac{ R_* t }{ \sqrt{ R_* t + R_{sky} n_{pix} t + (RN)^2 n_{pix} + DN\times n_{pix} t } }
    \end{equation}

    Let's make some assumptions and drop those small terms, e.g., 
    assume the time is long enough to drop RN and we don't care DN.

    So,
    \begin{equation}
        \begin{split}
            S/N &\propto \frac{ R_* }{ \sqrt{ R_* + R_{sky} n_{pix} } }\\
                &\propto \frac{ \sqrt{R_*} }{ \sqrt{ 1 + \frac{R_{sky}}{R_*} \mathrm{seeing}^2 } }\\
                &\propto \frac{1}{\sqrt{ 1 + \frac{R_{sky}}{R_*} \mathrm{seeing}^2 }}.
        \end{split}
    \end{equation}

    For our case, S/N roughly scales as 
    \begin{equation}
        \begin{split}
            (S/N \mid X=1.2, m_*=26, m_{sky}=20.3, MK)\\
                 \propto \frac{1}{\sqrt{ 1 + \frac{1}{4} \mathrm{seeing}^2 }}.
        \end{split}
    \end{equation}

    \item {\bf What is the exposure time to reach SN = 20?}

    Before solving $ S/N \mid X=1.2, m_{*}-26, m_{sky}=20.3 $, 
    we need to know $ R_{sky} \mid X=1.2 $.
    \begin{equation}
        (R_{sky} \mid X=1.2) = 1890 * 10^{ - \frac{20.3 - 20 + 0.12}{2.5} }
            \times 0.135^2
            = 23.395 \, \mathrm{e^- / s}.
    \end{equation}

    So the problem is just solving,
    \begin{equation}
        \begin{split}
            S/N = 20 
                &= \frac{ 6.74 t }{ \sqrt{ 6.74t + 23.395*49t + 5*49 + 1.2*49t } }\\
                &\sim \frac{7t}{\sqrt{7t + 1000t + 250 + 70t}}.
        \end{split}
    \end{equation}
    \begin{equation*}
        \Rightarrow 1077t + 250 \sim 49 / 400 t^2 \sim 1/8 t^2
    \end{equation*}
    \begin{equation*}
        \begin{split}
            \Rightarrow t^2 - 8600t - 2000 \sim 0\\
            \Rightarrow (t - 4300)^2 - 4300^2 \sim 2000\\
            \Rightarrow t \sim 4300 + \sqrt{ 4300^2 + 2000 } \sim 9000.
        \end{split}
    \end{equation*}

    So it's roughly $t \sim 9\,000$ seconds.  
    I carefully solved it numerically, 

    \lstset{caption={Solve t}}
    \lstset{label={lst:alg1}}
    \begin{lstlisting}[style = python]
    import sympy as sp

    # construct the signal to noise equation using lambda
    sn = lambda R_s, R_sky, RN, DN, npix, t : 
           R_s * t /
         ( R_s * t + R_sky * t * npix + RN**2 * npix + DN * t * npix )**(1/2)    
        
    # solve polynomial with sympy.solve
    expr = sn(6.74, 23.395, 5**(1/2), 1.2, 7**2, t)
    sp.solve(sp.Eq(expr, 20), t)
    \end{lstlisting} 

    and I found $t = 10\,671$ seconds.

    \item {\bf Find the exposure time on the HST.}
    
    I checked this filter and thought it's close enough to R band we used:
    \url{http://svo2.cab.inta-csic.es/svo/theory/fps3/index.php?id=HST/WFC3_UVIS2.F625W}
    
    After plugging those numbers, I got $t = 2059.6179 \sim 2\,000$ seconds. 
    
    \item {\bf Try out some parameters you are interested.}
    
    \begin{itemize}
        \item  0.4'' : $\sim 9\,500$ s
        \item 4''    : $\sim 700\,000$ s
        \item F600LP : $\sim 1\,000$ s
        \item 850LP  : $\sim 5\,000$ s
        \item Narrow 631N : $\sim 120\,000$ s
        \item Medium F21M : $\sim 5\,000$ s
    \end{itemize}

\end{enumerate}

\end{document}
